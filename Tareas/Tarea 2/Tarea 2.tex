\documentclass[12pt,a4paper]{article}
\usepackage[latin1]{inputenc}
\usepackage[spanish]{babel}
\usepackage{graphicx}
\usepackage{kpfonts}
\usepackage[left=2cm,right=2cm,top=2cm,bottom=2cm]{geometry}

\begin{document}
\title{Ingenieria mecatronica 4B \\
		Sistemas electronicos de interfaz}
\author{Angel Eraclio Briano Garcia 18311625\\Tarea 2 }
\date{24 de Septiembre del 2019}
$$\includegraphics[scale=1]{untitled.png}$$ 

\maketitle
\newpage
\section{Rectificador trifasico no controlados de media onda}

Hata los años 70, la mayor parte de la electronica de potencia se basaba en el uso de tiristores como interruptores controlados.
con SCRs se disenaban convertidores CC/CC, CC/CA, CA/CC y CA/CA.}


\section{Rectificador trifasico controlado de media onda.}
los diodos se sustituyen por SCRs y el disparo de los SCRs se retrasa un angulo respecto al cruce de las faces, su funcionamiento depende de su angulo y del caracter de la carga.

\section{Rectificadores trifasicos (sin tiristores).}
con el control adecuado de los interruptores se puede conseguir controlar la corriente por las entradas y la tension de salida.
con los interruptores bidireccionales en corriente y tension se puede conseguir que el rectificador funcione como inversor suministrando corriente senoidal en fase con la tension de entrada.
\section{Control manual de potencia electrica para un convertidor CA-CC (rectificador controlado).}
Se utiliza un tranformador de pulsos para aislar el circuito de disparo respecto a la tension de alimentacion de la carga.
la potencia en la carga se controla retrasando el disparo del triac respecto al cruce por cero de la tension de alimentacion.

\end{document}