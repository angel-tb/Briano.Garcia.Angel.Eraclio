\documentclass[12pt,a4paper]{article}
\usepackage[utf8]{inputenc}
\usepackage{amsmath}
\usepackage{amsfonts}
\usepackage{amssymb}
\usepackage{makeidx}
\usepackage{graphicx}
\usepackage[left=2cm,right=2cm,top=2cm,bottom=2cm]{geometry}

\begin{document}
\title{\textbf{Sistemas electronicos de interfaz\\EV 2.5. Arreglos de amplificadores de potencia\\Practica 6}}
\author{Josue Natanael Orozco Nevares 18311797\\Angel Eraclio Briano Garcia 18311625\\Ing. Mecatronica\\Grado 4B}
\date{8 de noviembre del 2019}
\maketitle

\begin{figure}[h!]
\centering
\includegraphics[width=10cm]{UPCDLZMDG5783-logo.png} 
\end{figure}
\newpage

\section{Introducciòn}
En esta practica conoceremos como es que funcionan los arreglos operacionales por medio de las simulaciones que se tiene que realizar.

\section{Objetivo}
Realizar las simulaciones correctamente y contestar lo que se te pida, en este caso es obtener la ganancia de cada tipo de circuito simulado las cuales son: \\• Inversor\\• No inversor\\• Sumador\\• Restador\\• Sumador restador

\section{Materiales}
• Laptop\\• Simulador de circuitos OrCAD

\section{Desarrollo}
\textbf{1-} Comenzaremos abriendo nuestro simulador para armar el primero de los circuitos el cual se trata de el \textbf{Inversor} y en este caso se representa en el siguiente circuito.

\begin{figure}[h!]
\centering
\includegraphics[scale=1]{Inversor1.png} 
\end{figure}
\newpage
Cuando se tenga el circuito comenzaremos a simularlo para ver el resultado de las ondas y asi poder calcular su ganancia con la siguiente formula R2/R1 y asì obtendremos nuestra ganancia la cual sera de 2.2.

\begin{figure}[h!]
\centering
\includegraphics[scale=1]{Inversor2.png} 
\end{figure}
\newpage
\textbf{2-} El siguiente circuito se trata de un \textbf{no inversor} y es representado por el siguiente circuito que se presenta a continuaciòn.

\begin{figure}[h!]
\centering
\includegraphics[scale=1]{NoInversor1.png} 
\end{figure}

Al igual que en el circuito pasado, tambien se colocaran las puntas de prueba en el mismo lugar, una en la entrada y otra en la salida, la manera de calcular la ganancia de un circuito \textbf{no inversor} es igual que la manera del \textbf{inversor} pero solo se le agrega un 1 a la formula y quedaria de la siguiente manera: R2/R1+1 y asi obtendremos la ganancia.

\begin{figure}[h!]
\centering
\includegraphics[scale=1]{NoInversor2.png} 
\end{figure}

\textbf{3-} El siguiente circuito es un \textbf{sumador} y el circuito simulado queda de la siguiente manera.

\begin{figure}[h!]
\centering
\includegraphics[scale=1]{Sumador.png} 
\end{figure}

Comenzaremos la simulaciòn para colocar nuestras puntas de prueba y asi lograr obtener las señales, la manera para obtener la ganancia de este circuito es algo mas compleja que las dos anteriores que ya vimos y se representa de la siguiente manera (V Rf/R1 + V Rf/R2 + V Rf/R3) para obtener asi la ganancia de este circuito \textbf{sumador}.


\textbf{4-} Este circuito se trata de un \textbf{restador} el cual se le asemeja en lo complejo del càlculo para obtener la ganancia y el circuito es el siguiente.

\begin{figure}[h!]
\centering
\includegraphics[scale=1]{Restador1.png} 
\end{figure}

Para ver las señales de entrada y de salida colocaremos las puntas como lo hemos estado haciendo para asi poder lograr obtener nuestra ganancia de manera visual y la manera de calcular la ganancia es (1 + Rf/R1)((Rx/R2+Rx)(V2-Rf/R1+Rf) (V1)) a pesar de que la formula es algo laboriosa en realidad funciona.

\begin{figure}[h!]
\centering
\includegraphics[scale=1]{Restador2.png} 
\end{figure}

\textbf{5-} El ultimo circuito se trata de un \textbf{sumador restador} y es el ultimo de los circuitos de esta practica ademas de ser el mas laborioso al momento de armarlo y la simulaciòn es la siguiente.

\begin{figure}[h!]
\centering
\includegraphics[scale=1]{SumadorRestador1.png} 
\end{figure}

En este circuito haremos un pequeño cambio al momento de colocar nuestras puntas ya que deberemos de colocar 3, dos en dos entradas distintas y una en la salida del amplificador como lo hemos estado haciendo en las demas practicas para poder ver asi nuestro resultado final.

\begin{figure}[h!]
\centering
\includegraphics[scale=1]{SumadorRestador2.png} 
\end{figure}

\section{Conclusiòn}
Quedo claro que la ganancia obtenida en cualquiera de los circuitos que realizamos aquì hace referencia muy clara a las señales electricas que estos emiten y esta se obtiene por medio de los amplificadores operacionales ademas de que la ganancia unicamente se puede llegar a detectar ya sea midiendola o calculandola por medio de las entradas y las salidas y estas frecuencias o se resultados se miden en belios o decibelios los cuales uno los asemeja a cuestiones sonoras, pero tambien se utilizan para la tensiòn o la potencia elèctrica.

\end{document}